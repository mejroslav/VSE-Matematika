\section*{Domácí úkol 1}
\textbf{Termín odevzdání:} na cvičení 7. nebo 8.10.2021.
\newline
\textbf{Zadání:} Cílem domácího úkolu je přesvědčit se, že platí \begin{align*}
    h(\mat A) = h(\mat A^T)  \:,
\end{align*}
kde $\mat A^T$ je transponovaná matice, která vznikne z matice $\mat A$ záměnou řádků za sloupce.

\begin{enumerate}
    \item \textbf{(0.5 bodu)} Sestavte matici $\mat A$ typu $4 \times 4$ takovou, že neobsahuje žádnou nulu a $h(\mat A) = 2$. 
    \item \textbf{(0.5 bodu)} Sestavte k ní transponovanou matici $\mat A^T$. Gaussovou eliminací určete její hodnost $h(\mat A^T)$. 
\end{enumerate}

Úkol odevzdejte, prosím, na papíře se svým jménem a časem cvičení, na které chodíte.