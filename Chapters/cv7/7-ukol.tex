\section*{Domácí úkol 4}
\textbf{Termín odevzdání:} na cvičení 11. nebo 12.11.2021.
\newline
\textbf{Zadání:} Cílem tohoto úkolu je pořádně procvičit počítání limit posloupností a funkcí.

\begin{itemize}
    \item \textbf{(0.5 bodu)} Vymyslete tři příklady limit typu
    \begin{align}
        \lim_{n \rightarrow \infty} \frac{P_k(n)}{P_l(n)} = \lim_{n \rightarrow \infty} \frac{\text{polynom stupně }k}{\text{polynom stupně }l} \:; \quad k,l \geq 5
    \end{align}
    tak, aby hodnota limity byla
    \begin{enumerate}
        \item $-\infty$,
        \item $2021$,
        \item $0$.
    \end{enumerate}
    Výpočtem ověřte.

    \item \textbf{(0.5 bodu)} Vymyslete tři příklady limit typu
    \begin{align}
        \lim_{x \rightarrow x_0+} \frac{P_k(x)}{P_l(x)} = \lim_{x \rightarrow x_0+} \frac{\text{polynom stupně }k}{\text{polynom stupně }l} \:; \quad k,l \geq 5 \:, x_0 \text{ je vlastní bod}
    \end{align}
    tak, aby hodnota limity byla
    \begin{enumerate}
        \item $+\infty$,
        \item $0$,
        \item $1$.
    \end{enumerate}

    \item \textbf{(0.5 bodu)} Vymyslete příklad limity funkce, při jejímž výpočtu použijete větu o dvou strážnících. \textbf{NEBO:} Vymyslete příklad limity složené funkce, která obsahuje funkci $\ln x$ jako vnější funkci a její limitní hodnota bude $0$.
    
    \item \textbf{(0.5 bodu)} Vymyslete příklad limity funkce typu
    \begin{align}
        \lim_{x \rightarrow \infty} \ln f(x) \cdot \arctg g(x) \:,
    \end{align}
    kde zvolíte funkce $f(x)$ a $g(x)$ vhodně tak, aby hodnota limity byla $-\frac{\pi}{2}$.
\end{itemize}