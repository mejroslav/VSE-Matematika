\section{Průběh funkce}



\subsection{Kuchařka pro vyšetřování průběhu funkce}

\begin{enumerate}

    \item Určíme definiční obor $D(f)$.
    
    \item Zkusíme zjistit, zda je funkce symetrická:
    \begin{itemize}
        \item Pro sudou funkci platí $f(x) = f(-x)$.
        \item Pro lichou funkci platí $-f(x) = - f(-x)$.
    \end{itemize}

    \item Spočítáme průsečíky funkce s osami $P_x$ a $P_y$:
    \begin{itemize}
        \item Bod $P_x$ nalezneme řešením rovnice
        $f(x)=0$.
        \item Bod $P_y$ nalezneme prostým dosazením bodu $0$ do funkce, tj. spočteme $f(0)$.
    \end{itemize}
     
    \item Určíme limity v bodech nespojitosti funkce a limity v bodech $\pm \infty$ (pokud existují).

    \item Vyšetříme lokální minima a maxima:
    \begin{itemize}
        \item Spočteme $f'(x)$, $f''(x)$ a najdeme stacionární body $x_0$, které odpovídají rovnici $f'(x_0) = 0$.
        \item Jestliže $f''(x_0) < 0$, pak má $f$ v $x_0$ lokální maximum.
        \item Jestliže $f''(x_0) > 0$, pak má $f$ v $x_0$ lokální minimum.
        \item Nalezneme funkční hodnoty $f(x_0)$, případně je mezi sebou porovnáme.
    \end{itemize}
    
    \item Nalezneme inflexní body řešením rovnice $f''(x_{\text{flex}}) = 0$.
    
    \item Určíme intervaly, na kterých $f'(x)$ a $f''(x)$ nemění znaménko.
    \begin{itemize}
        \item Na intervalu, kde platí $f'(x) > 0$, je $f$ rostoucí.
        \item Na intervalu, kde platí $f'(x) < 0$, je $f$ klesající.
        \item Na intervalu, kde platí $f''(x) > 0$, je $f$ konvexní.
        \item Na intervalu, kde platí $f''(x) < 0$, je $f$ konkávní.
    \end{itemize}

    \item Někdy můžeme vyšetřovat asymptotické chování funkce.
    \begin{itemize}
        \item Lineární asymptoty jsou přímky $u(x) = px+q$, kde 
        \begin{align*}
            p = \lim_{x \rightarrow \pm \infty} \frac{f(x)}{x} \quad \text{a} \quad q = \lim_{x \rightarrow \pm \infty} f(x) - px \:.
        \end{align*}
        Samozřejmě nemusí existovat.
        \item Můžeme také limitní chování porovnávat s mocninnými funkcemi typu $x^\alpha$, tj. nalézt takové $\alpha$, aby \begin{align*}
            \lim_{x \rightarrow \pm \infty} \frac{f(x)}{x^\alpha} \quad \text{byla vlastní a nenulová} \:.
        \end{align*}
        V takovém případě pak píšeme $f(x) = O(x^\alpha)$ pro $x \rightarrow \pm \infty$.
    \end{itemize}

    \item Nakreslíme graf funkce $f(x)$.
\end{enumerate}

\begin{example}
    Vyšetříme průběh funkce $f(x) = \frac{x^2-1}{x^2-5x+6}$.

\begin{enumerate}
    \item Definiční obor funkce: jmenovatel nesmí být nulový, tedy řešíme rovnici $x^2-5x +6 = 0$. Jejím řešením jsou body $3$ a $2$, tedy \begin{align*}
        D(f) = \R \setminus \set{3,2} \:.
    \end{align*}

    \item Funkce zjevně není ani symetrická, ani antisymetrická.
    \item Průsečík s osou x: řešíme rovnici $x^2-1 = 0$. Řešením jsou body $\pm 1$.
    Průsečík s osou y: stačí spočítat $f(0) = -\frac{1}{6} $.
    Průsečíky jsou tedy \begin{align*}
        P_x^{(1)} = [-1;0] \:, \quad P_x^{(2)} = [1;0] \:, \quad P_y = [0;-1/6] \:.
    \end{align*}
    \item Určíme limity funkce v $\pm \infty$:
    \begin{align*}
        \lim_{x \rightarrow + \infty} \frac{x^2-1}{x^2-5x+6} = 1 \:, \quad \lim_{x \rightarrow - \infty} \frac{x^2-1}{x^2-5x+6} = 1 \:.
    \end{align*}
    Také nesmíme zapomenout na limity v bodech nespojitosti:
    \begin{align*}
        \lim_{x \rightarrow 2_-} \frac{x^2-1}{x^2-5x+6} 
        =+ \infty \:, 
        \quad 
        \lim_{x \rightarrow 2_+} \frac{x^2-1}{x^2-5x+6} 
        = - \infty \:, \\
        \lim_{x \rightarrow 3_-} \frac{x^2-1}{x^2-5x+6} 
        = - \infty \:,
        \quad 
        \lim_{x \rightarrow 3_+} \frac{x^2-1}{x^2-5x+6} 
        = + \infty \:.
    \end{align*}
\end{enumerate}
\end{example}





