\section*{Definiční obory elementárních funkcí}

Při určování definičního oboru elementárních funkcí se v podstatě můžeme řídit jednoduchými zásadami.

\begin{enumerate}
    \item Nesmíme dělit nulou.
    \item Sudé odmocniny jsou definované pouze pro nezáporná čísla.
    \item Logaritmus je definovaný pouze pro kladná čísla.
    \item Speciální pozornost si zaslouží funkce $\tg x$, $\cotg x$, $\arcsin x $ a $\arccos x$.
\end{enumerate}

Kompletní přehled dává tabulka \ref{tab:funkce}. (Je víceméně potřeba umět ji nazpaměť.)

\begin{table}[H]
    \centering
    \begin{tabular}{|c|c|c|}
        \hline
        \textbf{funkce} & \textbf{definiční obor} & \textbf{obor hodnot} \\
        \hline
        $x^k$, $k$ je sudé              & $\R$          & $[0, \infty)$ \\ 
        $x^k$, $k$ je liché             & $\R$          & $\R$ \\ 
        $\sqrt[k]{x}$, $k$ je sudé      & $[0, \infty)$ & $[0, \infty)$ \\
        $\sqrt[k]{x}$, $k$ je liché     & $\R$          & $\R$ \\
        \hline
        $e^x$                           & $\R$          & $(0, \infty)$ \\
        $a^x$, $a>0$                    & $\R$          & $(0, \infty)$ \\
        $\ln x$                         & $(0, \infty)$ & $\R$ \\
        $\log_a x$, $a>0$               & $(0, \infty)$ & $\R$ \\
        \hline
        $\sin x$                        & $\R$          & $[0,1]$ \\
        $\cos x$                        & $\R$          & $[0,1]$ \\
        $\tg x$                         & $\R \setminus \set{\pi/2 + k \pi, k \in \Z}$ & $\R$ \\
        $\cotg x$                       & $\R \setminus \set{k \pi, k \in \Z}$  & $\R$ \\
        \hline
        $\arcsin x$                     & $[-1,1]$      & $[-\frac{\pi}{2}, \frac{\pi}{2}]$ \\
        $\arccos x$                     & $[-1,1]$      & $[0,\pi]$ \\
        $\arctg x$                      & $\R$          & $(-\frac{\pi}{2}, \frac{\pi}{2})$ \\
        $\arccotg x$                    & $\R$          & $(0,\pi)$ \\
         \hline
    \end{tabular}
    \caption{Tabulka elementárních funkcí.}
    \label{tab:funkce}
\end{table}

\section*{Příklady}

\begin{example}
    Určíme definiční obor funkce \begin{align}
        f(x) = \frac{1}{x-4} + \frac{1}{x+6} \:.
    \end{align}
    P1 nám říká, že nesmíme dělit nulou. To vede na podmínky $x \neq 4$ a $x \neq -6$. To jsou body, které musíme vyřadit z množiny reálných čísel, abychom získali definiční obor. Množinově to lze zapsat jako 
    \begin{align}
        D_f = \R \setminus \set{4,-6} \:,
    \end{align}
    kde symbol \uv{$\setminus$} značí množinový rozdíl. To samé můžeme napsat pomocí sjednocení intervalů 
    \begin{align}
        D_f = (-\infty,-6) \cup (-6,4) \cup (4,+\infty) \:.
    \end{align}
    Kulaté závorky značí otevřené intervaly, to znamená, že do nich krajní body nepatří.
\end{example}


\begin{example}
    Určíme definiční obor funkce \begin{align}
        g(x) = \frac{(x-2)(x-6)}{(x-4)(x^2-9)} \:.
    \end{align}
    Použijeme opět P1 a dostáváme rovnici \begin{align}
        (x-4)(x^2-9) = 0 \:.
    \end{align}
    Nyní využijeme toho, že \textit{součin čísel je roven nule právě tehdy, když alespoň jedno z nich je rovno nule}. Dostáváme tedy podmínky:
    \begin{align}
        x - 4 = 0 \implies x \neq 4 \\
        x^2 - 9 = 0 \implies x \neq \pm 3 \:.
    \end{align}
    Celkově $D_g = \R \setminus \set{-3,3,4}$.
\end{example}

\begin{example}
    Určíme definiční obor funkce \begin{align}
        h(x) = \sqrt{1 - \frac{6}{x+1}} \:.
    \end{align}
    P1 nám dává podmínku $x \neq -1$. P2 nám říká, že celý výraz pod odmocninou musí být nezáporný, tedy \begin{align}
        1 - \frac{6}{x+1} \geq 0 \:.
    \end{align}
    Tuto nerovnici můžeme vyřešit například tak, že převedeme oba členy na stejného jmenovatele
    \begin{align}
        \frac{x+1}{x+1} - \frac{6}{x+1} = \frac{x-5}{x+1} \geq 0
    \end{align}
    a poté využijeme toho, že \textit{podíl dvou čísel je nezáporný tehdy, když čitatel i jmenovatel budou buďto oba dva kladné nebo oba dva záporné}. Samozřejmě ale musíme vyloučit možnost $x=-1$. Takže \begin{align}
        [x-5 \leq 0] \bigwedge [x+1 > 0] \implies x \in [5,\infty) \\
        [x-5 \geq 0] \bigwedge [x+1 < 0] \implies x \in (-\infty,-1) 
    \end{align}
    Celkově $D_h = (-\infty, -1) \cup [5,\infty)$.
\end{example}

\begin{example}
    Určíme definiční obor funkce \begin{align}
        F(x) = \frac{\sqrt{4-\ln x}}{2^x - 16} \:.
    \end{align}
    P1 vede na rovnici $2^x - 16 = 0$. Tu můžeme snadno vyřešit, když si všimneme, že $16 = 2^4$, takže $x \neq 4$.
    P3 vede na podmínku $x>0$. Konečně P2 vede na nerovnici \begin{align}
        4 - \ln x \geq 0 \:.
    \end{align}
    Tu můžeme vyřešit tak, že logaritmus převedeme na jednu stranu rovnice
    \begin{align}
        4 \geq \ln x \:.
    \end{align}
    Nyní můžeme na rovnici \uv{zapůsobit exponenciálou}. Exponenciála je funkce prostá a rostoucí, nemění se tedy znaménko nerovnosti. Dostáváme \begin{align}
        \exp(4) = e^4 \geq \exp(\ln x) = x \:,
    \end{align}
    takže $x \in (-\infty, e^4]$.
    
    Celkově dostáváme $D_F = (0,4)\cup(4,e^4]$.
\end{example}

\begin{example}
    Určíme definiční obor funkce \begin{align}
        G(t) = \arcsin (\ln t) \:.
    \end{align}
    P3 nám říká, že $t>0$. Nyní se podíváme na pravidlo P4. To nám říká, že argument (\uv{vnitřek}) arkussinu musí být v mezích $[-1,1]$. Odtud dostáváme podmínku \begin{align}
        -1 \leq \ln t \leq +1 \:.
    \end{align}
    Tyto nerovnice můžeme vyřešit opět tak, že zapůsobíme exponenciálou:
    \begin{align}
        e^{-1} \leq t \leq e^1 \:.
    \end{align}
    Celkově tedy $D_G = [e^{-1},e]$.
\end{example}

\begin{example}
    Určíme definiční obor funkce \begin{align}
        y(x) = \ln \left( \frac{\pi}{6} - \arcsin x \right) \:.
    \end{align}
    P4 nám říká, že $x \in [-1,1]$. P3 nám dává nerovnici \begin{align}
        \frac{\pi}{6} - \arcsin x > 0 \:.
    \end{align}
    Takovou nerovnici opět vyřešíme tak, že arkussinus převedeme na druhou stranu rovnice
    \begin{align}
        \frac{\pi}{6} > \arcsin x
    \end{align}
    a na rovnici \uv{zapůsobíme} funkcí sinus. Ta je opět rostoucí, takže nezmění znaménko nerovnosti. Dostáváme
    \begin{align}
        \frac{1}{2} = \sin \left( \frac{\pi}{6} \right)  > \sin (\arcsin(x)) = x \:,
    \end{align}
    takže máme $x \in (-\infty, \frac{1}{2})$.

    Celkově $D_y = [-1,\frac{1}{2})$.
\end{example}

\begin{example}[Náročnější]
    Určíme definiční obor funkce \begin{align}
        T(z) = \ln \left(\ln(\ln z) \right) \:.
    \end{align}
    Aplikujeme pravidlo P3, postupovat budeme od funkce uvnitř. Dostaneme podmínky \begin{align}
        z > 0 \:, \quad \ln (z) > 0 \:, \quad \ln (\ln z) > 0 \:.
    \end{align}
    Druhá z nerovnic nám dává podmínku $z > 1$. Na třetí nerovnici aplikujeme exponenciálu a dostaneme \begin{align}
        \ln z > \exp 0 = 1 \:.
    \end{align}
    Nyní znova zapůsobíme exponenciálou a dostaneme \begin{align}
        z >  \exp 1 = e \:.
    \end{align}

    Takže $D_T = (e, \infty)$.
\end{example}

\begin{example}[S absolutní hodnotou]
    Určíme definiční obor funkce \begin{align}
        A(x) = \frac{1}{\sqrt{|x+2|-|x+4|}} \:.
    \end{align}
    P1 a P2 vedou na podmínku \begin{align}
        |x+2|-|x+4| > 0 \:.
    \end{align}
    Absolutních hodnot se zbavíme tak, že se podíváme na jednotlivé intervaly. Nulové body v absolutních hodnotách jsou $-2$ a $-4$. Dostáváme tak tři intervaly, na kterých budeme absolutní hodnoty řešit:
    \begin{table}[H]
        \centering
        \begin{tabular}{c|c|c|c}
            interval & $(-\infty,-4) $ & $(-4,-2)$ & $(-2,\infty)$ \\
            \hline
            $|x+2|$ & $-x-2$ & $-x-2$ & $x+2$ \\
            $|x+4|$ & $-x-4$ & $x+4$ & $x+4$ \\
            \hline
            $|x+2|-|x+4|$ & $(-x-2)-(-x-4)=2$ & $(-x-2)-(x+4)=-2x-6$ & $(x+2)-(x+4) = -2$
        \end{tabular}
    \end{table}
    Na intervalu $(-\infty,-4)$ se tedy suma absolutních hodnot chová jako konstanta $2$, která je zřejmě větší než nula. 
    
    Na intervalu $(-4,-2)$ musíme vyřešit nerovnici $-2x-6>0$, která vede na $x < -3$. 
    
    Na intervalu $(-2,\infty)$ už je chování zase konstantní, $-2<0$. 
    
    Celkově vyhovují pouze $x$ z intervalu $(-\infty,-4]$ a ještě z intervalu $[-4,-3)$. Takže $D_A = (-\infty,-3)$.
\end{example}

\subsection*{Závěrečné poznámky}

\begin{itemize}
    \item Časté chyby, na které je dobré dát zvláštní pozor:
    \begin{itemize}
        \item Záporná mocnina kladného čísla je stále kladné číslo! Například $2^{-4} = \frac{1}{2^4} = \frac{1}{16} > 0$.\newline To je rozdíl oproti $2^{-4} \neq -2^4 = -16$!
        \item Násobíme-li nerovnici záporným číslem, obrací se znaménko nerovnosti! Například nerovnici $-2x^2 > -4x$ můžeme vydělit $-2$ a dostaneme $x^2 < 2x$.
        \item Definiční obor musíme určit z původního výrazu, ještě předtím, než ho začneme dále upravovat. Tak například funkce \begin{align}
            f(x) = 1 \:, \quad g(x) = \frac{x(x+1)}{x(x+1)}
        \end{align}
        dávají pro všechna přípustná $x$ stejné hodnoty, ale definiční obor funkcí je různý! \newline $D_f = \R$, zatímco $D_g = \R \setminus \set{0,-1}$.
    \end{itemize}

\end{itemize}
