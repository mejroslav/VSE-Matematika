\section*{Cramerovo pravidlo}

Cramerovo pravidlo nám umožňuje řešit soustavu $n$ rovnic o $n$ neznámých
\begin{align}
    \mat A \vc x = \vc b \:, 
    \quad \text{kde}\quad 
    \vc x = \begin{pmatrix}
        x_1 \\ x_2 \\ \vdots \\ x_n
    \end{pmatrix} \:,\quad
    \vc b = \begin{pmatrix}
        b_1 \\ b_2 \\ \vdots \\ b_n
    \end{pmatrix} \:.
\end{align}
Je-li matice $\mat A$ regulární, pak má soustava právě jedno řešení. Jednotlivé neznámé $x_j$ (složky vektoru $\vc x$) lze napočítat vztahem
\begin{align}
    \boxed{ x_j = \frac{\det \mat A_j}{\det \mat A} }\:,
\end{align}
kde matice $\mat A_j$ vznikne tak, že na $j$-tý sloupec matice $\mat A$ dosadíme vektor pravých stran $\vc b$.

\begin{example}
    Pomocí Cramerova pravidla řešme soustavu 
    \begin{align}
        x_1 + 4 x_2 + x_3 = 3 \:, \\ 3 x_1 - x_2 - x_3 = 1 \:, \\ 2x_1 + x_2 + 2x_3 = 6 \:.
    \end{align}
    Sestavíme si matici soustavy a vektor pravých stran
    \begin{align}
        \mat A  = \begin{pmatrix}
            1 & 4 & 1 \\ 3 & -1 & -1 \\ 2 & 1 & 2
        \end{pmatrix} \:,
        \vc b = \begin{pmatrix}
            3 \\ 1 \\ 6
        \end{pmatrix} \:.
    \end{align}
    Nyní můžeme sestavit dílčí matice
    \begin{align}
        \mat A_1 = \begin{pmatrix}
            3 & 1 & 6 \\ 3 & -1 & -1 \\ 2 & 1 & 2
        \end{pmatrix}
        \:, \quad
        \mat A_2  = \begin{pmatrix}
            1 & 4 & 1 \\ 3 & 1 & 6 \\ 2 & 1 & 2
        \end{pmatrix} \:, \quad
        \mat A  = \begin{pmatrix}
            1 & 4 & 1 \\ 3 & -1 & -1 \\ 3 & 1 & 6
        \end{pmatrix} \:.
    \end{align}
    Spočítáme determinanty
    \begin{align}
        \det \mat A = -28 \:, \quad \det \mat A_1 = -28 \:, \quad \det \mat A_2 = 0 \:, \quad \det \mat A_3 = -56
    \end{align}
    a můžeme psát jednotlivé složky:
    \begin{align}
        x_1 = \frac{-28}{-28} = 1 \:, \quad x_2 = 0 \:, \quad x_3 = \frac{-56}{-28} = 2 \:,
    \end{align}
    nebo chceme-li vektorově
    \begin{align}
        \vc x = \begin{pmatrix}
            1 \\ 0 \\ 2
        \end{pmatrix} \:.
    \end{align}
\end{example}

\begin{itemize}
    \item Cramerovo pravidlo lze použít pouze v případě, že soustava má právě jedno řešení. Jinak bychom měli matici $\mat A$ singulární.
    \item Výpočetně není Cramerovo pravidlo příliš výhodné. Hodí se nám ale v situacích, kde máme velkou soustavu, ale potřebujeme znát pouze některé proměnné a zbylé ne. Pak se možná vyplatí spočítat pár determinantů namísto Gaussovy eliminace. Rovněž se Cramerovo pravidlo může hodit pro soustavy s parametrem.
\end{itemize}