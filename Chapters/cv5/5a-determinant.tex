\section*{Determinant}

Determinant je číslo definované pro čtvercovou matici, které nám pomůže určit, jestli je matice regulární nebo singulární. Označujeme jej symbolem $\det \mat A$ anebo $|\mat A|$.

Platí následující tvrzení:
\begin{align}
    \boxed{ \text{čtvercová matice } \mat A \text{ je singulární } \: \Longleftrightarrow \: \det \mat A = 0 } \:.
\end{align}

\subsection*{Jak ho spočítat?}

\begin{itemize}
    \item \textbf{řád 1:} matice $\mat A = (a_{11})$ má determinant $\det \mat A = a_{11}$.
    \item \textbf{řád 2:} matice $\mat A = \begin{pmatrix}
        a_{11} & a_{12} \\ a_{21} & a_{22}
    \end{pmatrix}$ má determinant $\det \mat A = a_{11} a_{22} - a_{12} a_{21}$. Takže od součinu prvků na hlavní diagonále odečítáme součin prvků na vedlejší diagonále.
    \item \textbf{řád 3:} matice $\mat A = \begin{pmatrix}
        a_{11} & a_{12} & a_{13} \\ a_{21} & a_{22} & a_{23} \\ a_{31} & a_{32} & a_{33}
    \end{pmatrix}$ má determinant, který je složen z šesti členů. Namísto obecného vzorečku si představíme mnemotechnickou pomůcku, podle které se počítá. Nazývá se \textbf{Sarrusovo pravidlo}:
    \begin{enumerate}
        \item Pod matici si napíšeme ještě jednou její první dva řádky.
        \item Tři součiny se znaménkem plus bereme podél hlavní diagonály.
        \item Tři součiny se znaménkem minus bereme podél vedlejší diagonály. 
    \end{enumerate}

    \begin{example}
        Spočteme \begin{align}
            \begin{vmatrix}
                3 & -2 & 6 \\
                1 & -1 & 1 \\
                0 & 4 & 1 
            \end{vmatrix} \:.
        \end{align}
        Napíšeme si schéma \begin{align}
            \begin{array}{ccc}
                | 3 & -2 & 6 |\\
                | 1 & -1 & 1 |\\
                | 0 & 4 & 1 | \\
                3 & -2 & 6 \\
                1 & -1 & 1
            \end{array}
        \end{align}
        a budeme počítat:
        \begin{align}
            \begin{vmatrix}
                3 & -2 & 6 \\
                1 & -1 & 1 \\
                0 & 4 & 1 
            \end{vmatrix} =& 3 \cdot (-1) \cdot 1 + 1 \cdot 4 \cdot 6 + 0 \cdot (-2) \cdot 1 - 0 \cdot (-1) \cdot 6 - 3 \cdot 4 \cdot 1 - 1 \cdot (-2) \cdot 1 
            = \\ =& -3 + 24 + 0 - 0 - 12 + 2 = 11 \:.
        \end{align}
    \end{example}
    
    \item \textbf{vyšší řády:} pro počítání determinantů matic vyšších řádů se používá nejčastěji tzv. \textbf{Laplaceův rozvoj}. Opět si napíšeme místo obecného vztahu princip, kterým se provádí:
    \begin{enumerate}
        \item Vybereme si určitý řádek nebo sloupec (nejčastěji takový, ve kterém je nejvíce nul).
        \item Elementy v řádku označíme znaménky. Prvek $a_{ij}$ dostane znaménko rovné $(-1)^{i+j}$, tzn. $+1$ je-li součet sloupce a řádku sudé číslo, $-1$, je-li liché.
        \item Sestavíme tzv. minory matice. Minor získáme vždy tak, že vynecháme právě zvolený řádek i sloupec.
        \item Spočítáme determinanty jednotlivých minorů.
        \item Každý determinant minoru vynásobíme prvkem, jehož řádek a sloupec jsme zrovna vynechali.
        \item Takto získané determinanty sečteme i s příslušnými znaménky.
    \end{enumerate}
    Na první pohled to vypadá složitě, ale není tomu tak.
    \begin{example}
        Spočítejme determinant
        \begin{align}
            \det \mat B = \left|\begin{matrix}
                4 & 2 & 1 & -2 \\
                1 & -3 & 1 & 0 \\
                2 & -1 & -4 & -1 \\
                1 & 0 & 0 & 1
                \end{matrix}\right| \:.
        \end{align}
        \begin{enumerate}
            \item Vybereme si řádek s nejvíce nulami, v našem případě čtvrtý.
            \item Každému elementu v řádku přiřadíme znaménko podle výše uvedeného pravidla. Například člen $a_{14}$ má součet $1+4=5$, což je liché číslo, takže mu přiřadíme $-1$. Dostaneme
            \begin{align}
                \begin{pmatrix}
                    + & - & + & - \\
                    - & + & - & + \\
                    + & - & + & - \\
                    \boldsymbol{-} & \boldsymbol{+} & \boldsymbol{-} & \boldsymbol{+}
                \end{pmatrix} \:.
            \end{align}
            \item Sestavíme minory, které budeme označovat symbolem $\mat M_i$. Vždy budeme vynechávat čtvrtý řádek. Jako první vynecháme první sloupec a dostaneme matici
            \begin{align}
                \mat M_1 = \left(\begin{matrix}
                    2 & 1 & -2 \\
                    -3 & 1 & 0 \\
                    -1 & -4 & -1
                    \end{matrix}\right) \:.
            \end{align}
            Pak vynecháme druhý sloupec a dostaneme matici
            \begin{align}
                \mat M_2 = \left(\begin{matrix}
                    4 & 1 & -2 \\
                    1 & 1 & 0 \\
                    2 & -4 & -1
                    \end{matrix}\right) \:.
            \end{align}
            Pak vynecháme třetí sloupec a dostaneme matici
            \begin{align}
                \mat M_3 = \left(\begin{matrix}
                4 & 2 & -2 \\
                1 & -3 & 0 \\
                2 & -1 & -1
                \end{matrix}\right) \:.
            \end{align}
            Nakonec vynecháme čtvrtý sloupec a dostaneme
            \begin{align}
                \mat M_4 = \left(\begin{matrix}
                    4 & 2 & 1 \\
                    1 & -3 & 1  \\
                    2 & -1 & -4
                \end{matrix}\right) \:.
            \end{align}

            \item Všem těmto minorům spočteme determinanty. Ve skutečnosti nás ale budou zajímat pouze $\mat M_1$ a $\mat M_4$, protože je nakonec budeme násobit daným prvkem, který je u $\mat M_2$ a $\mat M_3$ nula.
            \begin{align}
                \det \mat M_1 = -31 \:, \quad \det \mat M_4 = 69 \:.
            \end{align}

            \item Nyní můžeme psát:
            \begin{align}
                \det \mat B = (-1) \cdot (-31) + (+1) \cdot 0 + (-1) \cdot 0 + (+1) \cdot 69 = 31 + 69 = 100 \:.
            \end{align}
        \end{enumerate}

    \end{example}

    \begin{example}
        Určíme \begin{align}
            \det \mat K = \left|\begin{matrix}
                1 & 1 & 0 & 2 & 0 \\
                0 & 1 & 0 & -2 & 3 \\
                2 & 0 & -9 & -1 & 1 \\
                0 & 0 & 0 & 4 & 1 \\
                0 & 1 & 3 & 0 & 0
                \end{matrix}\right| \:.
        \end{align}
        Máme mnoho možností. Zvolme například rozvoj podle třetího sloupce. Takže máme schéma \begin{align}
            \begin{pmatrix}
            + & - & \boldsymbol{+} & - & + \\
            - & + & \boldsymbol{-} & + & - \\
            + & - & \boldsymbol{+} & - & + \\
            - & + & \boldsymbol{-} & + & - \\
            + & - & \boldsymbol{+} & - & +
            \end{pmatrix}
        \end{align}
        a počítáme
        \begin{align}
            \det \mat K = (+1) \cdot (-9) \cdot
            \begin{vmatrix}
                1 & 1 & 2 & 0 \\
                0 & 1 & -2 & 3 \\
                0 & 0 & 4 & 1 \\
                0 & 1 & 0 & 0
            \end{vmatrix}
            + (+1) \cdot 3 \cdot
            \begin{vmatrix}
                1 & 1  & 2 & 0 \\
                0 & 1  & -2 & 3 \\
                2 & 0  & -1 & 1 \\
                0 & 0  & 4 & 1
            \end{vmatrix}
            = -9 \cdot \det \mat P_1 + 3 \cdot \det \mat P_2 \:.
        \end{align}
        Musíme spočítat oba dva determinanty. Pro $\det \mat P_1$ můžeme ihned použít rozvoj podle prvního sloupce:
        \begin{align}
            \det \mat P_1 = 1 \cdot
            \begin{vmatrix}
                 1 & -2 & 3 \\
                 0 & 4 & 1 \\
                 1 & 0 & 0
            \end{vmatrix}
            =  -14 \:.
        \end{align}
        Pro $\det \mat P_2$ můžeme použít rozvoj podle druhého sloupce:
        \begin{align}
            \det \mat P_2 = (-1) \cdot 1 \cdot \begin{vmatrix}
                0 & -2 & 3 \\
                2 & -1 & 1 \\
                0 & 4 & 1
            \end{vmatrix}
            + (+1) \cdot 1 \cdot
            \begin{vmatrix}
                1 & 2 & 0 \\
                2 & -1 & 1 \\
                0 & 4 & 1
            \end{vmatrix}
            =
            -28 + (-9) = -37 \:.
        \end{align}
        Celkově
        \begin{align}
            \det \mat K = (-9) \cdot (-14) + 3 \cdot (-37) = 126 - 111 = 15 \:.
        \end{align}
    \end{example}
\end{itemize}

\subsection*{Determinant pomocí řádkových (sloupcových úprav)}

Výpočet determinantu můžeme provádět i tak, že provádíme podobné řádkové (nebo sloupcové) úpravy v matici. Využíváme přitom tvrzení, že \textbf{determinant horní trojúhelníkové matice je roven součinu prvků na diagonále}. Je však třeba dát si obrovský pozor, neboť řádkové úpravy nemůžeme provádět \uv{bezmyšlenkovitě}, jako např. u Gaussovy eliminace.
\begin{itemize}
    \item Při prohození dvou řádků (sloupců) se změní znaménko determinantu.
    \item Při násobení řádku (sloupce) číslem $\alpha$ musíme determinant vydělit číslem $\alpha$.
    \item Libovolné dva řádky můžeme sečíst, determinant pak zůstává stejný.
\end{itemize}

\begin{example}
    Spočteme \begin{align}
        \begin{vmatrix}
            4 & -1 & 2 \\ 1 & 2 & 0 \\ -2 & -1 & 1
        \end{vmatrix}
    \end{align}
    pomocí řádkových úprav.

    \begin{align}
        \begin{vmatrix}
            4 & -1 & 2 \\ 1 & 2 & 0 \\ -2 & -1 & 1
        \end{vmatrix}
        =&
        \left( - \frac{1}{4} \right) \cdot \frac{1}{2} \cdot \begin{vmatrix}
            4 & -1 & 2 \\ -4 & -8 & 0 \\ -4 & -2 & 2
        \end{vmatrix}
        =
        - \frac{1}{8} \begin{vmatrix}
            4 & -1 & 2 \\ 0 & -9 & 2 \\ 0 & -3 & 4
        \end{vmatrix}
        = \\ =&
        - \frac{1}{8} \cdot \left( -\frac{1}{3}\right) \begin{vmatrix}
            4 & -1 & 2 \\ 0 & -9 & 2 \\ 0 & 9 & -12
        \end{vmatrix}
        =
        \frac{1}{24}
        \begin{vmatrix}
            4 & -1 & 2 \\ 0 & -9 & 2 \\ 0 & 0 & -10
        \end{vmatrix}
        = \frac{1}{24} \cdot 4 \cdot (-9) \cdot (-10) = 15 \:.
    \end{align}
\end{example}