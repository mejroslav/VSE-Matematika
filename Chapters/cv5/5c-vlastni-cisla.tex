\section*{Vlastní čísla}

Zjistili jsme, že násobení vektoru maticí je poměrně početně náročná záležitost. Nyní bychom chtěli nalézt speciální vektory, pro které takové maticové násobení bude jednoduchou záležitostí.

Mějme matici $\mat A$ typu $n \times n$. Mezi všemi vektory $\vc v \in V_n$ zkusme nalézt speciální $\vc u$, pro které se násobení maticí $\mat A$ chová jako obyčejné násobení číslem $\lambda$. Vyřešme proto rovnici
\begin{align}
    \mat A \vc u = \lambda \vc u \:.
\end{align}
Tato rovnice představuje soustavu lineárních rovnic
\begin{align}
    (\mat A - \lambda \mat J) \vc u = \vc 0 \:.
\end{align}
Jedná se o homogenní soustavu. O ní víme, že má určitě jedno triviální řešení $\vc u = \vc 0$. Nulový vektor ale není žádný zázrak. Zkusme najít nějaké další.

Představme si, že matice $\mat A - \lambda \mat J$ je regulární. Pak bychom mohli rovnici vynásobit inverzí a dostali bychom
\begin{align}
    \vc u = (\mat A - \lambda \mat J)^{-1} \vc 0 = \vc 0 \:,
\end{align}
protože násobení nulového vektoru libovolnou maticí je stále nulový vektor. My ale chceme hledat nenulové vektory. Proto požadujeme, aby $\mat A - \lambda \mat J$ byla singulární matice!

Požadavek na singularitu vede na nulovost determinantu:
\begin{align}
    \det (\mat A - \lambda \mat J) = 0 \:.
\end{align}
Na tuto rovnici se musíme koukat následovně: máme zadánu matici $\mat A$ a hledáme speciální čísla $\lambda$, které této rovnici vyhovují.
Rovnici říkáme \textbf{vlastní (charakteristická) rovnice} pro matici $\mat A$ a sepciálním číslům $\lambda$ říkáme \textbf{vlastní (charakteristická) čísla} matice $\mat A$. (Výraz vlastní, charakteristický pochází z německého \uv{eigen} - vlastnit. V angličtině se vlastním číslům říká \uv{eigenvalues}.) Vektorům $\vc u$, které splňují rovnici $\mat A \vc u =\lambda \vc u$ říkáme vlastní vektory. Výpočtem vlastních vektorů se nebudeme zabývat. Výpočet vlastních čísel si ovšem ukážeme.

\begin{example}
    Určíme vlastní čísla matice \begin{align}
        \mat D = \begin{pmatrix}
            7 & 5 \\ 6 & 8
        \end{pmatrix} \:.
    \end{align}
    Sestavíme matici
    \begin{align}
        \mat D - \lambda \mat J = \begin{pmatrix}
            7-\lambda & 5 \\ 6 & 8-\lambda
        \end{pmatrix}
    \end{align}
    a spočteme její determinant:
    \begin{align}
        \det (\mat D - \lambda \mat J) = (7-\lambda)(8-\lambda) - 5 \cdot 6 = 56 - 7\lambda - 8 \lambda + \lambda^2 - 30 = \lambda^2 - 15 \lambda + 26 \:.
    \end{align}
    Potřebujeme tedy vyřešit rovnici
    \begin{align}
        \lambda^2 - 15 \lambda + 26 = 0 \:.
    \end{align}
    Jejím řešením jsou
    \begin{align}
        \lambda_{1,2} =  \frac{15 \pm \sqrt{15^2-4\cdot 26}}{2} = \frac{15 \pm \sqrt{225-104}}{2} = \frac{15 \pm \sqrt{121}}{2} = \frac{15 \pm 11}{2} = 2, 13 \:.
    \end{align}
    Nalezli jsme dvě vlastní čísla $\lambda_1 = 2$ a $\lambda_2 = 13$. Každému z těchto čísel přísluší nějaký vlastní vektor, ten ale hledat nebudeme.
\end{example}

\begin{itemize}
    \item Matice $n \times n$ má $n$ vlastních čísel. Ne všechna ale musejí být různá, stejně třeba jako kořeny kvadratické rovnice. Některá vlastní čísla se mohou objevit vícekrát, říkáme o nich, že mají \textbf{algebraickou násobnost} větší než 1.
    \item \textbf{Jakmile je matice singulární, je $0$ jejím vlastním číslem.} To plyne ihned z charakteristické rovnice: $\det (\mat A ) = 0$, tedy $\det (\mat A - 0 \cdot \mat J) = 0$.
    \item O matici řekneme, že je symetrická, jestliže $\mat A = \mat A^T$. \textbf{Vlastní čísla symetrické matice jsou vždy reálná.} Pokud matice není symetrická, může mít i \textbf{komplexní vlastní čísla}.
\end{itemize}

\subsection*{Dodatek: vlastní vektory}

\begin{example}
    Jak bychom v předchozím příkladu hledali vlastní vektory? Pro každé vlastní číslo $\lambda$ máme rovnici $ (\mat D - \lambda \mat J) \vc u = 0$.
    
    Pro vlastní číslo $\lambda_1 = 2$ tedy máme rovnici $(\mat D - 2 \mat J) \vc u = 0$. Označíme-li $\vc = \begin{pmatrix}
        u_1 \\ u_2
    \end{pmatrix}$, pak řešíme rovnici
    \begin{align}
        \begin{pmatrix}
            7-2 & 5 \\ 6 & 8-2
        \end{pmatrix}
        \begin{pmatrix}
            u_1 \\ u_2
        \end{pmatrix}
        =
        \begin{pmatrix}
            0 \\ 0
        \end{pmatrix} \:.
    \end{align}
    Tomu odpovídá soustava rovnic
    \begin{align}
        \left(\begin{array}{cc|c}
            5 & 5 & 0 \\
            6 & 6 & 0
        \end{array} \right) \:,
    \end{align}
    stačí tedy vzít $u_2 = t \in \R$ a $u_1 = - u_2 = -t$. Vlastní vektor
    \begin{align}
        \vc u_1 = \begin{pmatrix}
            -1 \\ 1
        \end{pmatrix} \cdot t
    \end{align}
    je tedy libovolný násobek vektoru $(-1,1)$. Skutečně, můžeme si ověřit, že maticové násobení pro tento vektor je skutečně stejné jako násobení číslem:
    \begin{align}
        \mat D \vc u_1 = \begin{pmatrix}
            7 & 5 \\ 6 & 8
        \end{pmatrix}
        \begin{pmatrix}
            -t \\ t
        \end{pmatrix}
        =
        \begin{pmatrix}
            -7t+5t \\ -6t+8t
        \end{pmatrix}
        =
        \begin{pmatrix}
            -2t \\ 2t
        \end{pmatrix}
        = 2 \vc u_1 \:.
    \end{align}

    Obdobně bychom našli vlastní vektor $\vc u_2$ k vlastnímu číslu $\lambda_2 = 13$.
\end{example}