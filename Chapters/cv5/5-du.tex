\section*{Domácí úkol 3}
\textbf{Termín odevzdání:} na cvičení 4. nebo 5.11.2021.
\newline
\textbf{Zadání:} V první části domácího úkolu si vyzkoušejte zkratku pro počítání inverzní matice typu $2 \times 2$. Mějme matici \begin{align}
    \mat A = \begin{pmatrix}
        a & b \\ c & d
    \end{pmatrix} \:.
\end{align}
Pak inverzní matice má tvar
\begin{align}
    \mat A^{-1} = \frac{1}{ad-bc} \begin{pmatrix}
        d & -b \\ -c & a 
    \end{pmatrix} \:.
\end{align}
Povšimněme si, jak ji získáme:
\begin{enumerate}
    \item Na hlavní diagonále prohodíme prvky.
    \item Na vedlejší diagonále otočíme znaménka.
    \item Celou matici vydělíme determinantem původní matice.
\end{enumerate}

\begin{itemize}
    \item \textbf{(0.5 bodu)} K maticím
    \begin{align}
        \mat K = \begin{pmatrix}
            4 & 2 \\ 3 & 2
        \end{pmatrix} \:,\quad
        \mat L = \begin{pmatrix}
            10 & 3 \\ 3& 1
        \end{pmatrix} \:,\quad
        \mat M = \begin{pmatrix}
            -9 & 2 \\ -4 & 1
        \end{pmatrix}
    \end{align}
    najděte inverzní, nejdříve pomocí Jordanovy metody, potom vyzkoušejte nové pravidlo.
\end{itemize}

Ve druhé části domácího úkolu zkuste najít vztah mezi vlastními čísly matice $\mat A$ a $\mat A^{-1}$.
\begin{itemize}
    \item \textbf{(0.5 bodu)} Najděte vlastní čísla $\lambda$ matice \begin{align}
        \mat R = \begin{pmatrix}
            1 & -1 \\ 1 & 3
        \end{pmatrix} \:.
    \end{align}
    Poté najděte matici $\mat R^{-1}$ a vlastní čísla $\kappa$ této matice. Platí mezi $\kappa$ a $\lambda$ nějaký vztah?
\end{itemize}

\textbf{Poznámka:} \begin{itemize}
    \item Zkratku pro počítání inverze 2 krát 2 nepoužívejte v zápočtových testech ani u zkoušky! Můžete ji využít pouze pro rychlou kontrolu.
\end{itemize}