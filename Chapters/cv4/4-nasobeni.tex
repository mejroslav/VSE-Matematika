\section*{Násobení matic}

Matice $\mat A$ typu $n \times m$ a $\mat B$ typu $m \times k$ lze spolu vynásobit. Výsledná matice $\mat A \mat B$ je typu $n \times k$ a získáme ji pomocí pravidla \uv{řádek na sloupec}.

\begin{example}[Násobení pomocí tabulky]
    Nechť
    \begin{align}
        \mat A = \begin{pmatrix}
            2 & 4 & -1 & 0 \\
            1 & 0 & 3 & -2
        \end{pmatrix}
        \:, \quad
        \mat B = \begin{pmatrix}
            -1 & 1 & 2 \\
            3 & 0 & 0 \\
            1 & -2 & -1 \\
            -4 & 1 & 0 \\
        \end{pmatrix} \:.
    \end{align}
    Matice můžeme vynásobit pomocí pravidla řádek na sloupec. Výsledná matice bude zřejmě typu $2 \times 3$. Zápis, který může být užitečný, je sepsat si do tabulky \textbf{vlevo první matici} a \textbf{nahoru druhou matici}:
    \begin{align}
        \begin{array}{rrrr|rrr}
            &&&&-1 & 1 & 2 \\
            &&&&3 & 0 & 0 \\
            &&&&1 & -2 & -1 \\
            &&&&-4 & 1 & 0 \\
            \hline
            2 & 4 & -1 & 0 &     9 & 4 & 5\\
            1 & 0 & 3 & -2 &     10 & -7 & -1\\
        \end{array}
    \end{align}
    Například člen $1,1$ jsme získali součtem součinů: 
    \begin{align}
        2 \cdot (-1) + 4 \cdot 3 + (-1) \cdot 1 + 0 \cdot (-4) = 9 \:.
    \end{align}
    Takže \begin{align}
        \mat A \mat B = \begin{pmatrix}
            9 & 4 & 5 \\
            10 & -7 & -5
        \end{pmatrix} \:.
    \end{align}
    Matice v opačném pořadí vůbec vynásobit nelze, protože by neseděly rozměry.
\end{example}

\begin{example}[Nekomutativita matic]
    Ačkoli například čtvercové matice $\mat A$ a $\mat B$ můžeme násobit v obou směrech, neplatí, že by $\mat A \mat B$ a $\mat B \mat A$ byly stejné matice!
    Například \begin{align}
        \begin{pmatrix}
            1 & -1 \\
            0 & 2
        \end{pmatrix}
        \begin{pmatrix}
            -1 & -1 \\
            1 & 4
        \end{pmatrix}
        =
        \begin{pmatrix}
            -2 & -5 \\
            2 & 8
        \end{pmatrix} \:, \quad
        \begin{pmatrix}
            -1 & -1 \\
            1 & 4
        \end{pmatrix}
        \begin{pmatrix}
            1 & -1 \\
            0 & 2
        \end{pmatrix}
        =
        \begin{pmatrix}
            -1 & -1\\
            1 & 7\\
        \end{pmatrix} \:.
    \end{align}

    Přesvědčte se sami, že násobit v obou směrech lze i matice typu $m \times n$ a $n \times m$. Rozměr výsledné matice ale bude pokaždé jiný!

    \underline{Musíme proto přísně rozlišovat mezi násobením matic zleva a zprava.}
\end{example}

Pro násobení matic platí
\begin{itemize}
    \item \textbf{nekomutativita}: $\mat A \mat B \neq \mat B \mat A$
    \item \textbf{asociativita}: $\mat A (\mat B \mat C) = (\mat A \mat B) \mat C$
    \item \textbf{distributivita} (zprava a zleva): $(\mat A + \mat B) \mat C = \mat A \mat C + \mat B \mat C$, $\mat A(\mat B + \mat C) = \mat A \mat B + \mat A \mat C$
\end{itemize}

\section*{Inverzní matice}

Čtvercové matice typu $n \times n$ označujeme jako matice řádu $n$.

Definujeme \textbf{jednotkovou matici} $\mat J$ řádu $n$ (v literatuře se pro ni používá též značení $\mat E$, $\mat I$, $\mat E_n$, $\mat{Id}$) jako čtvercovou matici řádu $n$, která má na hlavní diagonále samé jedničky a jinde nuly. Je-li $\mat A$ čtvercová matice řádu $n$, pak zjevně platí $\mat A \mat J = \mat J \mat A = \mat A$.

\textbf{Inverzní matice} $\mat A^{-1}$ je čtvercová matice stejného řádu $n$, která splňuje $\mat A \mat A^{-1} = \mat A^{-1} \mat A = \mat J$. Pokud inverzní matice existuje, je definována jednoznačně. Není ale definována pro všechny čtvercové matice. 

Řekneme, že matice $\mat A$ řádu $n$ je \textbf{regulární}, jestliže $h(\mat A) = n$, v opačném případě řekneme, že je \textbf{singulární}. Pouze pro regulární matice existuje matice inverzní.

\subsection*{Hledání inverzní matice Jordanovou metodou}

Pro výpočet inverzní matice lze použít metodu, kterou jsme zmiňovali u řešení soustavy lineárních rovnic. Matici $\mat A$ rozšíříme o jednotkovou matici $\mat J$ a pomocí elementárních řádkových úprav se ji pokusíme převést na tvar, kde bude na levé straně vystupovat jednotková matice $\mat J$. Na pravé straně pak získáme inverzní matici $\mat A^{-1}$.
\begin{align}
    (\mat A | \mat J) \rightarrow \text{eřú} \rightarrow (\mat J | \mat A^{-1})
\end{align}

\begin{example}
    Určíme inverzní matici k \begin{align}
        \mat B = \begin{pmatrix}
            1 & -1 & 0 \\
            2 & 0 & 1 \\
            -1 & -1 & 4
        \end{pmatrix} \:.
    \end{align}
    Tato matice má $h(\mat B ) = 3$, je proto regulární a inverzní matice k ní existuje.
    Napíšeme si tvar \begin{align}
        (\mat B | \mat J) = \left(\begin{array}{rrr|rrr}
            1 & -1 & 0 & 1 & 0 & 0\\
            2 & 0 & 1 & 0 & 1 & 0\\
            -1 & -1 & 4 & 0 & 0 & 1
        \end{array}\right) \:.
    \end{align}
    Nyní začneme provádět eřú. Nejprve se snažíme vynulovat členy pod diagonálou, stejně jako u Gaussovy eliminace. Začneme standartně prvním sloupcem a pokračovat budeme druhým sloupcem.
    \begin{align}
        \left(\begin{array}{rrr|rrr}
            1 & -1 & 0 & 1 & 0 & 0\\
            2 & 0 & 1 & 0 & 1 & 0\\
            -1 & -1 & 4 & 0 & 0 & 1
        \end{array}\right) 
        \sim
        \left(\begin{array}{rrr|rrr}
            1 & -1 & 0 & 1 & 0 & 0\\
            0 & 2 & 1 & -2 & 1 & 0\\
            0 & -2 & 4 & 1 & 0 & 1
        \end{array}\right)
        \sim
        \left(\begin{array}{rrr|rrr}
            1 & -1 & 0 & 1 & 0 & 0\\
            0 & 2 & 1 & -2 & 1 & 0\\
            0 & 0 & 5 & -1 & 1 & 1
        \end{array}\right) \:.
    \end{align}
    Nyní chceme vynulovat členy nad diagonálou. Začneme posledním sloupcem a pokračovat budeme prostředním. Poslední řádek rovnou vydělíme pěti.
    \begin{align}
        \left(\begin{array}{rrr|rrr}
            1 & -1 & 0 & 1 & 0 & 0\\
            0 & 2 & 1 & -2 & 1 & 0\\
            0 & 0 & 5 & -1 & 1 & 1
        \end{array}\right)
        \sim&
        \left(\begin{array}{rrr|rrr}
            1 & -1 & 0 & 1 & 0 & 0\\
            0 & 2 & 0 & -2+\frac{1}{5} & 1-\frac{1}{5} & -\frac{1}{5}\\
            0 & 0 & 1 & -\frac{1}{5} & \frac{1}{5} & \frac{1}{5}
        \end{array}\right)
        \sim
        \left(\begin{array}{rrr|rrr}
            1 & -1 & 0 & 1 & 0 & 0\\
            0 & 1 & 0 & -\frac{9}{10} & \frac{4}{10} & -\frac{1}{10}\\
            0 & 0 & 1 & -\frac{1}{5} & \frac{1}{5} & \frac{1}{5}
        \end{array}\right)
        \sim \\ \sim&
        \left(\begin{array}{rrr|rrr}
            1 & 0 & 0 & 1-\frac{9}{10} & \frac{4}{10} & -\frac{1}{10}\\
            0 & 1 & 0 & -\frac{9}{10} & \frac{4}{10} & -\frac{1}{10}\\
            0 & 0 & 1 & -\frac{1}{5} & \frac{1}{5} & \frac{1}{5}
        \end{array}\right)
        \sim
        \left(\begin{array}{rrr|rrr}
            1 & 0 & 0 & \frac{1}{10} & \frac{4}{10} & -\frac{1}{10}\\
            0 & 1 & 0 & -\frac{9}{10} & \frac{4}{10} & -\frac{1}{10}\\
            0 & 0 & 1 & -\frac{2}{10} & \frac{2}{10} & \frac{2}{10}
        \end{array}\right) \:.
    \end{align}
    Nalevo máme jednotkovou matici. Napravo jsme získali matici inverzní:
    \begin{align}
        \mat B^{-1} = \begin{pmatrix}
            \frac{1}{10} & \frac{4}{10} & -\frac{1}{10}\\
            -\frac{9}{10} & \frac{4}{10} & -\frac{1}{10}\\
            -\frac{2}{10} & \frac{2}{10} & \frac{2}{10}
        \end{pmatrix}
        = \frac{1}{10} \begin{pmatrix}
            1 & 4 & -1 \\
            -9 & 4 & -1 \\
            -2 & 2 & 2
        \end{pmatrix}
    \end{align}

    Můžeme se přesvědčit o tom, že je to správná inverzní matice:
    \begin{align}
        \mat B \mat B^{-1} = \frac{1}{10} \begin{pmatrix}
            1 & -1 & 0 \\
            2 & 0 & 1 \\
            -1 & -1 & 4
        \end{pmatrix} \begin{pmatrix}
            1 & 4 & -1 \\
            -9 & 4 & -1 \\
            -2 & 2 & 2
        \end{pmatrix}
        =
        \frac{1}{10}
        \begin{pmatrix}
            10 & 0 & 0 \\ 0 & 10 & 0 \\ 0 & 0 & 10
        \end{pmatrix}
        = \mat J
    \end{align}
    a obdobně v opačném pořadí.
\end{example}

\begin{example}
    Matice \begin{align}
        \mat M = \begin{pmatrix}
            4 & 2 \\
            -2 & -1
        \end{pmatrix}
    \end{align}
    nemá k sobě inverzní, protože $h(\mat M) = 1$. Pokud bychom se pokusili hledat ji Jordanovou metodou, narazili bychom na potíže:
    \begin{align}
        \left(\begin{array}{rr|rr}
            4 & 2 & 1 & 0\\
            -2 & -1 & 0 & 1
        \end{array}\right)
        \sim
        \left(\begin{array}{rr|rr}
            4 & 2 & 1 & 0\\
            0 & 0 & 1 & 2
        \end{array}\right) \:.
    \end{align}
    Na levé straně se nám žádným způsobem nepovede vynulovat dvojku nad diagonálou.
\end{example}

\subsection*{Poznámky}

\begin{itemize}
    \item Asi nejsnáze se určuje regularita/singularita matice pomocí determinantu.
    \item K hledání inverzní matice řádu 2 se používá často triku, který je v českých zemích pojmenován \uv{Čihákovo pravidlo} na počest profesora Čiháka, který přednášel dlouhá léta na MATFYZu. Seznámíme se s ním v jednom pozdějším domácím úkolu.
    \item Další vlastnost maticové algebry je, že $(\mat A \mat B)^{-1} \neq \mat A^{-1} \mat B^{-1}$. Ve skutečnosti $(\mat A \mat B)^{-1} = \mat B^{-1} \mat A^{-1}$, což si ověříme rovněž v domácím úkolu.
    \item Stejně tak platí $(\mat A \mat B)^T = \mat B^T \mat A^T$.
\end{itemize}