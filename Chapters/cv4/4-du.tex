\section*{Domácí úkol 2}
\textbf{Termín odevzdání:} na cvičení 21. nebo 22.10.2021.
\newline
\textbf{Zadání:} Cílem domácího úkolu je přesvědčit se o identitě tvaru \begin{align*}
    (\mat A \mat B)^{-1} = \mat B^{-1} \mat A^{-1} \:.
\end{align*}

\begin{itemize}
    \item \textbf{(0.25 bodu)} Vymyslete dvě \uv{netriviální} regulární matice $\mat A$, $\mat B$ řádu $2$.
    \item \textbf{(0.25 bodu)} Spočtěte jejich inverzní matice $\mat A^{-1}$, $\mat B^{-1}$.
    \item \textbf{(0.25 bodu)} Spočtěte součin $\mat C = \mat B^{-1} \mat A^{-1}$.
    \item \textbf{(0.25 bodu)} Spočtěte součin $\mat D = \mat A \mat B$ a k této matici sestrojte inverzní $\mat D^{-1}$. \newline(Výsledné matice by si měly odpovídat: $\mat C = \mat D^{-1}$)
\end{itemize}

\textbf{Poznámka:} Z identity je vidět, že součin dvou regulárních matic bude opět regulární matice (protože k takovému součinu existuje inverzní matice). Povšimněte si také, že $(\mat A \mat B)^{-1} \neq \mat A^{-1} \mat B^{-1}$.